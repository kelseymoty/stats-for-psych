% Options for packages loaded elsewhere
\PassOptionsToPackage{unicode}{hyperref}
\PassOptionsToPackage{hyphens}{url}
%
\documentclass[
]{book}
\usepackage{lmodern}
\usepackage{amssymb,amsmath}
\usepackage{ifxetex,ifluatex}
\ifnum 0\ifxetex 1\fi\ifluatex 1\fi=0 % if pdftex
  \usepackage[T1]{fontenc}
  \usepackage[utf8]{inputenc}
  \usepackage{textcomp} % provide euro and other symbols
\else % if luatex or xetex
  \usepackage{unicode-math}
  \defaultfontfeatures{Scale=MatchLowercase}
  \defaultfontfeatures[\rmfamily]{Ligatures=TeX,Scale=1}
\fi
% Use upquote if available, for straight quotes in verbatim environments
\IfFileExists{upquote.sty}{\usepackage{upquote}}{}
\IfFileExists{microtype.sty}{% use microtype if available
  \usepackage[]{microtype}
  \UseMicrotypeSet[protrusion]{basicmath} % disable protrusion for tt fonts
}{}
\makeatletter
\@ifundefined{KOMAClassName}{% if non-KOMA class
  \IfFileExists{parskip.sty}{%
    \usepackage{parskip}
  }{% else
    \setlength{\parindent}{0pt}
    \setlength{\parskip}{6pt plus 2pt minus 1pt}}
}{% if KOMA class
  \KOMAoptions{parskip=half}}
\makeatother
\usepackage{xcolor}
\IfFileExists{xurl.sty}{\usepackage{xurl}}{} % add URL line breaks if available
\IfFileExists{bookmark.sty}{\usepackage{bookmark}}{\usepackage{hyperref}}
\hypersetup{
  pdftitle={Statistics for Behavioral Sciences},
  hidelinks,
  pdfcreator={LaTeX via pandoc}}
\urlstyle{same} % disable monospaced font for URLs
\usepackage{longtable,booktabs}
% Correct order of tables after \paragraph or \subparagraph
\usepackage{etoolbox}
\makeatletter
\patchcmd\longtable{\par}{\if@noskipsec\mbox{}\fi\par}{}{}
\makeatother
% Allow footnotes in longtable head/foot
\IfFileExists{footnotehyper.sty}{\usepackage{footnotehyper}}{\usepackage{footnote}}
\makesavenoteenv{longtable}
\usepackage{graphicx}
\makeatletter
\def\maxwidth{\ifdim\Gin@nat@width>\linewidth\linewidth\else\Gin@nat@width\fi}
\def\maxheight{\ifdim\Gin@nat@height>\textheight\textheight\else\Gin@nat@height\fi}
\makeatother
% Scale images if necessary, so that they will not overflow the page
% margins by default, and it is still possible to overwrite the defaults
% using explicit options in \includegraphics[width, height, ...]{}
\setkeys{Gin}{width=\maxwidth,height=\maxheight,keepaspectratio}
% Set default figure placement to htbp
\makeatletter
\def\fps@figure{htbp}
\makeatother
\setlength{\emergencystretch}{3em} % prevent overfull lines
\providecommand{\tightlist}{%
  \setlength{\itemsep}{0pt}\setlength{\parskip}{0pt}}
\setcounter{secnumdepth}{5}
\usepackage{booktabs}
\usepackage{amsthm}
\makeatletter
\def\thm@space@setup{%
  \thm@preskip=8pt plus 2pt minus 4pt
  \thm@postskip=\thm@preskip
}
\makeatother
\usepackage[]{natbib}
\bibliographystyle{apalike}

\title{Statistics for Behavioral Sciences}
\author{}
\date{\vspace{-2.5em}Summer 2021}

\begin{document}
\maketitle

{
\setcounter{tocdepth}{1}
\tableofcontents
}
\hypertarget{syllabus}{%
\chapter{Syllabus}\label{syllabus}}

\hypertarget{instructors}{%
\section{Instructors}\label{instructors}}

\textbf{Professor}: Kelsey Moty\\
\textbf{Email}: moty at nyu dot edu\\
\textbf{Office Hours}: By appointment (schedule via \href{https://calendly.com/kelseymoty/stats-office-hours}{Calendly} or by email, if none of the available slots work for you)\\
\textbf{Office}: The zoom world

\textbf{Teaching Assistant}: Ben Stillerman\\
\textbf{Email}: ben.stillerman at nyu dot edu\\
\textbf{Office Hours}: By appointment (schedule via email)

\hypertarget{place-and-time}{%
\section{Place and Time}\label{place-and-time}}

Zoom (see zoom link in Brightspace)\\
M/T/W 9:00 - 11:45 am (generally with an intermission halfway through)

\hypertarget{course-overview}{%
\section{Course Overview}\label{course-overview}}

The goal of this class is for you to become statistically literate. I think of this as driver's ed for statistics. Most of you won't end up being researchers or statisticians in the same way that most of you won't end up being auto mechanics or engineers for GM. You still benefit from knowing what a car is and how to operate it. Similarly, you will benefit from knowing what statistics is and a little bit about how to use it. For those of you who do want to go on to use statistics professionally, this will give you a strong foundation.

By the end of the class, you will be able to:

\begin{enumerate}
\def\labelenumi{\arabic{enumi}.}
\tightlist
\item
  Recognize different kinds of variables and data.
\item
  Understand most standard charts and graphs.
\item
  Read and understand statistical reports.
\item
  Conduct basic analyses of your own using statistical software.
\end{enumerate}

\hypertarget{textbooks}{%
\section{Textbooks}\label{textbooks}}

We're going to use a couple of different textbooks. All are optional but also free and useful, so I gotta suggest, take a look.

Cohen, B.H. (2013). Explaining Psychological Statistics (4th Ed). Wiley. --- This textbook is freely available
through NYU. \href{https://ebookcentral.proquest.com/lib/nyulibrary-ebooks/detail.action?docID=1563061}{Click this link} \textgreater{} then click ``Download Book''. You'll need to create a quick ProQuest account (name/email/pw) and download Adobe Digital Editions, which are both free.

\href{https://r4ds.had.co.nz/}{R for Data Science} --- It's free! It's online! It will teach you R! Good if you want to learn how to code or are interested in data science.

\href{https://learningstatisticswithr.com/}{Learning Statistics with R} and \href{https://learnstatswithjasp.com/}{Learning Statistics with JASP} -- Also free and online! They will teach you R and JASP! Extra points for the engaging way in which Danielle Navarro writes.

\hypertarget{statistical-software}{%
\subsection{Statistical Software}\label{statistical-software}}

Note that part of the class will be learning to use statistical software for data analysis. Specifically, R and JASP.

JASP is available for \href{https://jasp-stats.org/download/}{free download here}.

If you plan to use R, you will want to \href{https://cran.r-project.org/}{download R} as well as \href{https://www.rstudio.com/products/rstudio/download/}{RStudio}; both are free. Alternatively, you can make a \href{https://www.rstudio.com/products/cloud/}{free RStudio Cloud account} to get started with R. Using the cloud version ensures that everything has been properly installed and you'll have no issues when trying to load packages.

We will be teaching these skills (using R and JASP) in the class. However, if you find that you need extra assistance (particularly in learning R / they don't appear to have support for JASP), Bobst library provides software-specific tutorials as well as statistical consultants who are familiar with R. For more details, see \href{https://guides.nyu.edu/c.php?g=277138\&p=1847146}{their website}.

\hypertarget{assignments}{%
\section{Assignments}\label{assignments}}

\hypertarget{exams-50-using-only-your-highest-exam-grade}{%
\subsection{Exams (50\%, using only your highest exam grade)}\label{exams-50-using-only-your-highest-exam-grade}}

Rather than multiple exams that cover different portions of the material, you will take a final exam at the end of every week. The exams will have \textbf{nearly identical formats}, differing only in the particulars. \textbf{All exams are final exams}, meaning they cover \textbf{all of the material covered in the whole course}. This means you will have a final exam at the end of Week 1, at the end of Week 2, at the end of Week 3, and so on\ldots{}

Since these are all final exams, most of you probably won't do very well on the first exam. But that's ok, because:

Your exam grade will be \textbf{entirely} based on your \textbf{best exam}. Other exam grades will \textbf{not contribute at all}. If you get a 90\% on the third final, it doesn't matter how you did on the first two.

This format has some very specific benefits:

\begin{itemize}
\tightlist
\item
  \textbf{Low Anxiety:} Students who are able to get a good grade on one of the early tests will be able to worry about things other than cramming for the next test.
\item
  \textbf{Safety Net}: Each exam offers a new chance to set a minimum threshold for your grade. Once you get a 85 on one exam, you can rest easy that your grade won't go any lower.
\item
  \textbf{Transparency}: No more need for, ``what will be on the test?'' Once you have taken the first final, you will know (approximately) the format of all the other finals.
\item
  \textbf{Context}: Seeing all the material at once will allow you to begin building a tapestry of ideas in your head. You will never be blindsided by new material you are expected to know. Once you've seen one final exam you will have a good sense of all of them. Being exposed to all the material early on will help you learn it later.
\item
  \textbf{Good Incentives}: I want to reward you if you fully understand the material quickly. Students who succeed will be rewarded with more freedom. No one who has mastered the material should be forced to go through the motions. If you get a grade you're happy with, you can choose to skip the rest of the exams with no downside.
\item
  \textbf{Feedback}: You will be able to tell what skills you have mastered and which you need to work on. This will allow you to spend your study time wisely.
\end{itemize}

Exams will be run through the survey software Qualtrics. You will have \textbf{45 minutes to complete the exam} after the survey has been opened, so please be sure to not start the exam until you are able to complete the whole exam. These exams are open note, but must be completed by yourself (no working on exams with other people!).

\hypertarget{data-exploration-30-15-per-research-report}{%
\subsection{Data Exploration (30\%, 15\% per research report)}\label{data-exploration-30-15-per-research-report}}

This is an opportunity for you to gain some experience conducting analyses on your own.

In a folder on the Classes website there are a number of text files and matching csv files (a spreadsheet filetype). Each text file describes a (hypothetical) scenario, and the matching .csv includes the related data.

For a research report, you pick a dataset and conduct the appropriate analyses on it with the statistical software of your choice (or by hand if you really want to, I guess). In this class, we will be covering how to use R and JASP, but you are free to use other software if you wish.

You will not be told exactly which test(s) to run. Instead, you will determine this for yourself based on the scenario in the text file. The end product should be a document in which you describe the analyses you conducted, report the test results, and draw reasonable conclusions.

The report has the following requirements:

\begin{itemize}
\tightlist
\item
  One page long. Longer submissions won't be accepted. I want you to learn to be concise!
\item
  Twelve point font, 1.5-line spacing.
\item
  Report your statistical analyses in (approximately) APA format. We need to be able to understand it, but formatting isn't the point of the assignment.
\item
  Some sort of graph or plot
\item
  Discuss, at least briefly, your conclusions: say what the test results mean for the (hypothetical) study. Also mention if you notice anything strange or have any concerns about the results.
\end{itemize}

When you are done with a report, submit as a PDF on Brightspace to receive a grade.

You are required to do two of these research reports at some point during the class. You can do these two reports at any point during the class, but since you won't be taught how to do most analyses until about halfway through, most of you will probably want to do these assignments during the second half of the course.

If you are not satisfied with your grade on one or both of your research reports, you may do a replacement research report on a new dataset. You can only do a replacement report once for each research report. We want you to be able to learn from your mistakes but also I don't want you each doing 10 of these.

Essentially this means you can do up to four of these and your grade will be based on your best two. Obviously if you're happy with your first two, you don't need to do four. Please don't complete more than 4, the extras won't be graded.

\hypertarget{participation-20}{%
\subsection{Participation (20\%)}\label{participation-20}}

Measuring participation in an online class that doesn't require attendance is a bit tricky, but I also hope that you will also be actively engaging with the material as it is presented.

The best way I can measure participation (at least in my opinion) given these constraints is having you submit a short reflection of what was discussed in lecture that day.

\textbf{What should these reflections be about?} Really anything.

Maybe there's a topic that wasn't clear to you and would like some clarification on it. Maybe you found a connection between something we discussed today and something we addressed in a previous lecture. Maybe you found something interesting and want more details about it. Or maybe you want to provide some constructive feedback in how I presented something.

I'll also ask a number of questions during lecture to encourage participation (and may provide some questions as prompts in the reflection submission box). You can write a response to one of these questions in your reflection if you can't think of anything else.

\textbf{What should these reflections look like?} Nothing too formal and nothing particularly long. I don't want to put a specific minimum or maximum length because in the end, I see these reflections as an opportunity for myself to gauge where everyone is with the material (am I going too fast or too slow, etc.). A 2- to 4-sentence paragraph is generally an appropriate length. You won't be graded on grammar (incomplete sentences are ok); as long as a I can easily figure out what you are talking about, you're good.

\textbf{When are these due?} Within 36 hours after the lecture ends, but I'd recommend submitting these immediately after you watched/attended lecture. I'm giving a little extra time for you to complete these for a couple of reasons:

\begin{itemize}
\tightlist
\item
  Some of you are watching these lectures after the fact, so I want there to be enough time for the lecture to load to Brightspace (Zoom can take a couple plus hours for this happen) and for you to watch the lecture
\item
  But also for those attending live, some students may need some time for lectures to consolidate in their minds before they're able to reflect on what they learned. Hopefully this extra time gives students the time they need
\end{itemize}

\textbf{How many of these do I have to do?} One per lecture, but you can miss up to 2 and still get full credit. So you need to do 15 reflections for the 17 lectures.

\hypertarget{extra-credit-research-participation-up-to-2}{%
\subsection{Extra Credit: Research Participation (up to 2\%)}\label{extra-credit-research-participation-up-to-2}}

Many psychology studies are run at research universities and recruit undergraduates as participants. If you would like, you may participate in a research study for up to 2\% extra credit on your final grade. You will earn 0.5\% extra credit per half hour of research participation (i.e., 2 hours of participation for the full 2\% extra credit).

You can sign up for research experiments on Sona (\url{http://nyu-psych.sonasystems.com}), the psychology department's online experiment management system. To start signing up for studies, first request a Sona account at \url{http://nyu-psych.sonasystems.com}. Your login information will be sent to your NYU e-mail address within 24 hours. If you are under 18 years of age, please note that you will need a parent or guardian to sign a Parental Consent form prior to each study you complete.

Please read the \href{https://as.nyu.edu/content/dam/nyu-as/psychology/documents/research/research-requirements/AdvancedStudentGuide_Rev2017.pdf}{Advanced Psychology Student Guide} for more details and step-by-step instructions for creating a Sona account and signing up for research studies. If you are under 18 years of age, please note that you will need a parent or guardian to sign a Parental Consent form prior to each study you complete. Any questions/concerns about the research requirement or Sona should be directed to the Subject Pool Administrator, Brenda Woodford-Febres (\href{mailto:brenda.woodford@nyu.edu}{\nolinkurl{brenda.woodford@nyu.edu}}).

The deadline for research participation is August 13th. If you are unable to participate in research participation and want another opportunity to earn this extra credit, please reach out by Monday, July 26th to coordinate an alternate assignment.

\hypertarget{grading}{%
\subsection{Grading}\label{grading}}

It is each student's responsibility to monitor their grades online and report any discrepancies to the grader within one week of the contested assignment. DO NOT wait until the end of the term to check your grade. The grading scale is as follows:

\begin{longtable}[]{@{}llll@{}}
\toprule
\endhead
A = 93 - 100 & B = 83 - 86 & C = 73 - 76 & D = 63 - 66\tabularnewline
A- = 90 - 92 & B- = 80 - 82 & C- = 70 - 72 & D- = 60 - 62\tabularnewline
B+ = 87 - 89 & C+ = 77 - 79 & D+ = 67 - 69 & F \textless{} 60\tabularnewline
\bottomrule
\end{longtable}

For questions about Exam or Participation grades, contact Kelsey. For questions about Research Report grades, contact Ben.

\hypertarget{schedule}{%
\section{Schedule}\label{schedule}}

This is the schedule for the semester. This page will be frequently updated based on how we progress (plus I'll add links to extra readings and resources), and everything is subject to change. But this should give you a gist of the general topics that will be covered in this class.

If there is something in particular you would like covered in this class (that falls within the purview of an intro stats course) and you do not see it on the syllabus, please reach out to me.

\begin{longtable}[]{@{}llll@{}}
\toprule
\begin{minipage}[b]{0.09\columnwidth}\raggedright
\textbf{Week 1}\strut
\end{minipage} & \begin{minipage}[b]{0.32\columnwidth}\raggedright
\textbf{Part I}\strut
\end{minipage} & \begin{minipage}[b]{0.29\columnwidth}\raggedright
\textbf{Part II}\strut
\end{minipage} & \begin{minipage}[b]{0.18\columnwidth}\raggedright
\textbf{Readings (Optional)}\strut
\end{minipage}\tabularnewline
\midrule
\endhead
\begin{minipage}[t]{0.09\columnwidth}\raggedright
July 6\strut
\end{minipage} & \begin{minipage}[t]{0.32\columnwidth}\raggedright
Welcome\strut
\end{minipage} & \begin{minipage}[t]{0.29\columnwidth}\raggedright
What is statistics?\strut
\end{minipage} & \begin{minipage}[t]{0.18\columnwidth}\raggedright
\href{https://learningstatisticswithr.com/book/why-do-we-learn-statistics.html}{Why do we learn stats?}\strut
\end{minipage}\tabularnewline
\begin{minipage}[t]{0.09\columnwidth}\raggedright
July 7\strut
\end{minipage} & \begin{minipage}[t]{0.32\columnwidth}\raggedright
Intro to Data/Descriptive Stats\strut
\end{minipage} & \begin{minipage}[t]{0.29\columnwidth}\raggedright
Intro to Inferential Stats\strut
\end{minipage} & \begin{minipage}[t]{0.18\columnwidth}\raggedright
\strut
\end{minipage}\tabularnewline
\begin{minipage}[t]{0.09\columnwidth}\raggedright
\textbf{Week 2}\strut
\end{minipage} & \begin{minipage}[t]{0.32\columnwidth}\raggedright
\strut
\end{minipage} & \begin{minipage}[t]{0.29\columnwidth}\raggedright
\strut
\end{minipage} & \begin{minipage}[t]{0.18\columnwidth}\raggedright
\strut
\end{minipage}\tabularnewline
\begin{minipage}[t]{0.09\columnwidth}\raggedright
July 12\strut
\end{minipage} & \begin{minipage}[t]{0.32\columnwidth}\raggedright
Finishing up intro to Inferential Stats\strut
\end{minipage} & \begin{minipage}[t]{0.29\columnwidth}\raggedright
Data: Items \& Variables\strut
\end{minipage} & \begin{minipage}[t]{0.18\columnwidth}\raggedright
\href{https://en.wikipedia.org/wiki/Wide_and_narrow_data}{Long vs wide data}\strut
\end{minipage}\tabularnewline
\begin{minipage}[t]{0.09\columnwidth}\raggedright
July 13\strut
\end{minipage} & \begin{minipage}[t]{0.32\columnwidth}\raggedright
Data: Data Types\strut
\end{minipage} & \begin{minipage}[t]{0.29\columnwidth}\raggedright
Intro to JASP\strut
\end{minipage} & \begin{minipage}[t]{0.18\columnwidth}\raggedright
\href{https://learningstatisticswithr.com/book/studydesign.html\#scales}{Scales of measurement} / \href{https://tomfaulkenberry.github.io/JASPbook/chapters/chapter3.pdf}{Intro to JASP}\strut
\end{minipage}\tabularnewline
\begin{minipage}[t]{0.09\columnwidth}\raggedright
July 14\strut
\end{minipage} & \begin{minipage}[t]{0.32\columnwidth}\raggedright
Data: Outliers\strut
\end{minipage} & \begin{minipage}[t]{0.29\columnwidth}\raggedright
Intro to R\strut
\end{minipage} & \begin{minipage}[t]{0.18\columnwidth}\raggedright
\href{https://scholarworks.umass.edu/cgi/viewcontent.cgi?article=1139\&context=pare}{Outliers} / \href{https://learningstatisticswithr.com/book/introR.html}{Intro to R}\strut
\end{minipage}\tabularnewline
\begin{minipage}[t]{0.09\columnwidth}\raggedright
\textbf{Week 3}\strut
\end{minipage} & \begin{minipage}[t]{0.32\columnwidth}\raggedright
\strut
\end{minipage} & \begin{minipage}[t]{0.29\columnwidth}\raggedright
\strut
\end{minipage} & \begin{minipage}[t]{0.18\columnwidth}\raggedright
\strut
\end{minipage}\tabularnewline
\begin{minipage}[t]{0.09\columnwidth}\raggedright
July 19\strut
\end{minipage} & \begin{minipage}[t]{0.32\columnwidth}\raggedright
Descriptives: Central Tendency \& Dispersion\strut
\end{minipage} & \begin{minipage}[t]{0.29\columnwidth}\raggedright
Descriptives: Plots 1\strut
\end{minipage} & \begin{minipage}[t]{0.18\columnwidth}\raggedright
\href{https://learningstatisticswithr.com/book/descriptives.html\#centraltendency}{Central tendency} / \href{https://socvis.co/lookatdata.html\#lookatdata}{Why look at data?}\strut
\end{minipage}\tabularnewline
\begin{minipage}[t]{0.09\columnwidth}\raggedright
July 20\strut
\end{minipage} & \begin{minipage}[t]{0.32\columnwidth}\raggedright
Descriptives: Relation\strut
\end{minipage} & \begin{minipage}[t]{0.29\columnwidth}\raggedright
Descriptives: Plots 2\strut
\end{minipage} & \begin{minipage}[t]{0.18\columnwidth}\raggedright
\href{https://learningstatisticswithr.com/book/descriptives.html\#var}{Variability} / \href{https://learningstatisticswithr.com/book/graphics.html}{Plots}\strut
\end{minipage}\tabularnewline
\begin{minipage}[t]{0.09\columnwidth}\raggedright
July 22\strut
\end{minipage} & \begin{minipage}[t]{0.32\columnwidth}\raggedright
Inferential: Samples \& Populations\strut
\end{minipage} & \begin{minipage}[t]{0.29\columnwidth}\raggedright
Inferential: Tests Review\strut
\end{minipage} & \begin{minipage}[t]{0.18\columnwidth}\raggedright
\href{https://learningstatisticswithr.com/lsr-0.6.pdf}{Samples and populations: See Chap 10}\strut
\end{minipage}\tabularnewline
\begin{minipage}[t]{0.09\columnwidth}\raggedright
\textbf{Week 4}\strut
\end{minipage} & \begin{minipage}[t]{0.32\columnwidth}\raggedright
\strut
\end{minipage} & \begin{minipage}[t]{0.29\columnwidth}\raggedright
\strut
\end{minipage} & \begin{minipage}[t]{0.18\columnwidth}\raggedright
\strut
\end{minipage}\tabularnewline
\begin{minipage}[t]{0.09\columnwidth}\raggedright
July 26\strut
\end{minipage} & \begin{minipage}[t]{0.32\columnwidth}\raggedright
Inferential: \emph{p}-values and NHST\strut
\end{minipage} & \begin{minipage}[t]{0.29\columnwidth}\raggedright
Inferential: Power\strut
\end{minipage} & \begin{minipage}[t]{0.18\columnwidth}\raggedright
\strut
\end{minipage}\tabularnewline
\begin{minipage}[t]{0.09\columnwidth}\raggedright
July 27\strut
\end{minipage} & \begin{minipage}[t]{0.32\columnwidth}\raggedright
Inferential: One Sample \emph{t}-test\strut
\end{minipage} & \begin{minipage}[t]{0.29\columnwidth}\raggedright
Inferential: One-way \(\chi^2\)\strut
\end{minipage} & \begin{minipage}[t]{0.18\columnwidth}\raggedright
\strut
\end{minipage}\tabularnewline
\begin{minipage}[t]{0.09\columnwidth}\raggedright
July 28\strut
\end{minipage} & \begin{minipage}[t]{0.32\columnwidth}\raggedright
Inferential: Two Sample \emph{t}-test\strut
\end{minipage} & \begin{minipage}[t]{0.29\columnwidth}\raggedright
Open Science\strut
\end{minipage} & \begin{minipage}[t]{0.18\columnwidth}\raggedright
\strut
\end{minipage}\tabularnewline
\begin{minipage}[t]{0.09\columnwidth}\raggedright
\textbf{Week 5}\strut
\end{minipage} & \begin{minipage}[t]{0.32\columnwidth}\raggedright
\strut
\end{minipage} & \begin{minipage}[t]{0.29\columnwidth}\raggedright
\strut
\end{minipage} & \begin{minipage}[t]{0.18\columnwidth}\raggedright
\strut
\end{minipage}\tabularnewline
\begin{minipage}[t]{0.09\columnwidth}\raggedright
August 2\strut
\end{minipage} & \begin{minipage}[t]{0.32\columnwidth}\raggedright
Inferential: Correlation\strut
\end{minipage} & \begin{minipage}[t]{0.29\columnwidth}\raggedright
Inferential: Two-way \(\chi^2\)\strut
\end{minipage} & \begin{minipage}[t]{0.18\columnwidth}\raggedright
\strut
\end{minipage}\tabularnewline
\begin{minipage}[t]{0.09\columnwidth}\raggedright
August 3\strut
\end{minipage} & \begin{minipage}[t]{0.32\columnwidth}\raggedright
Inferential: Simple Regression\strut
\end{minipage} & \begin{minipage}[t]{0.29\columnwidth}\raggedright
Inferential: One-way ANOVA\strut
\end{minipage} & \begin{minipage}[t]{0.18\columnwidth}\raggedright
\strut
\end{minipage}\tabularnewline
\begin{minipage}[t]{0.09\columnwidth}\raggedright
August 4\strut
\end{minipage} & \begin{minipage}[t]{0.32\columnwidth}\raggedright
Inferential: Multiple Regression\strut
\end{minipage} & \begin{minipage}[t]{0.29\columnwidth}\raggedright
Inferential: Two-way ANOVA\strut
\end{minipage} & \begin{minipage}[t]{0.18\columnwidth}\raggedright
\strut
\end{minipage}\tabularnewline
\begin{minipage}[t]{0.09\columnwidth}\raggedright
\textbf{Week 6}\strut
\end{minipage} & \begin{minipage}[t]{0.32\columnwidth}\raggedright
\strut
\end{minipage} & \begin{minipage}[t]{0.29\columnwidth}\raggedright
\strut
\end{minipage} & \begin{minipage}[t]{0.18\columnwidth}\raggedright
\strut
\end{minipage}\tabularnewline
\begin{minipage}[t]{0.09\columnwidth}\raggedright
August 9\strut
\end{minipage} & \begin{minipage}[t]{0.32\columnwidth}\raggedright
Inferential: \emph{t}-tests under the hood\strut
\end{minipage} & \begin{minipage}[t]{0.29\columnwidth}\raggedright
Inferential: Correlation under the hood\strut
\end{minipage} & \begin{minipage}[t]{0.18\columnwidth}\raggedright
\strut
\end{minipage}\tabularnewline
\begin{minipage}[t]{0.09\columnwidth}\raggedright
August 10\strut
\end{minipage} & \begin{minipage}[t]{0.32\columnwidth}\raggedright
Inferential: \(\chi^2\) under the hood\strut
\end{minipage} & \begin{minipage}[t]{0.29\columnwidth}\raggedright
Inferential: Regression under the hood\strut
\end{minipage} & \begin{minipage}[t]{0.18\columnwidth}\raggedright
\strut
\end{minipage}\tabularnewline
\begin{minipage}[t]{0.09\columnwidth}\raggedright
August 11\strut
\end{minipage} & \begin{minipage}[t]{0.32\columnwidth}\raggedright
Inferential: ANOVA under the hood\strut
\end{minipage} & \begin{minipage}[t]{0.29\columnwidth}\raggedright
Advanced Topics / Review / Flex Day\strut
\end{minipage} & \begin{minipage}[t]{0.18\columnwidth}\raggedright
\strut
\end{minipage}\tabularnewline
\bottomrule
\end{longtable}

\hypertarget{policies}{%
\section{Policies}\label{policies}}

\hypertarget{attendance}{%
\subsection{Attendance}\label{attendance}}

Attendance is not required, but highly encouraged. We only meet for 6 weeks (a total of 17 classes), and classes build cumulatively off each other. If you are unable to attend class, all lectures will be recorded and made available on NYU Brightspace.

\hypertarget{honor-code-and-plagarism}{%
\subsection{Honor code and plagarism}\label{honor-code-and-plagarism}}

No form of academic dishonesty will be tolerated. Suspicions of academic dishonesty during exams or on writing assignments will be reported to and adjudicated by the Associate Dean for Students. If a case of academic dishonesty is confirmed, it will result in a penalty that is at least as severe as receiving a 0 on the assignment or quiz. Academic dishonesty includes all forms of plagiarism. Plagiarism includes (but is not limited to): copying or paraphrasing from someone else's work (another student, an online source, or a journal article), turning in someone else's work as your own, or presenting someone else's ideas (a student, online source, or scholar) as your own.

Most importantly, cheating only really serves to cheat yourself out of a learning opportunity. There are many opportunities to make up for a less-than-ideal grade, so don't let the possibility of a bad grade tempt you to cheat.

We will do our best to make clear the line between academic dishonesty and collaborative learning. But if you have any questions about where this line is, it is your responsibility to ask the instructor.

\hypertarget{accomodation}{%
\subsection{Accomodation}\label{accomodation}}

Any student requesting academic accommodations needs to notify the instructor before the third day of class. All discussions will remain confidential. Students with disabilities should also contact Disability Support Services in the \href{https://www.nyu.edu/students/communities-and-groups/student-accessibility.html}{Moses Center for Students with Disabilities}.

\hypertarget{extra-help}{%
\subsection{Extra Help}\label{extra-help}}

If you start to struggle at any time during this course, please let us know as early as possible. We are happy to help and make accommodations (especially given the extraordinary circumstances!) but it's a lot easier to do so early on than it is to change things after the fact. If you come to us at the end of the class there may be very little we can do.

The \href{https://www.nyu.edu/students/academic-services/undergraduate-advisement/academic-resource-center/tutoring-and-learning.html}{University Learning Center} has free peer tutoring and will be holding drop-in peer tutoring hours this summer, including weekly group session in support of statistics. Please see their website for their schedule.

\hypertarget{late-assignments}{%
\subsection{Late Assignments}\label{late-assignments}}

Due to the condensed nature of this course, late assignments cannot be accepted. However, the nature of the grading for this course provides many opportunities to make up for missed assignments. You can miss up to 2 reflection `papers' and still receive full credit for participation. Similarly, only your highest final exam will count toward your grade.

If you are unable to make an assignment deadline, please reach out to us before the assignment is due. However, if you are in the hospital with a serious illness, it's okay to reach out after the deadline when you feel a bit better (yes, this has happened, and I promise you your health and well-being is more important than this class).

\hypertarget{religious-observances}{%
\subsection{Religious observances}\label{religious-observances}}

Some students may wish to take part in religious observances that occur during this academic term. If you have a religious observance which conflicts with your participation in the course, please meet with me by the end of the second week of the term to discuss appropriate accommodations.

\hypertarget{pandemic-and-just-the-general-nature-of-life}{%
\subsection{Pandemic and just the general nature of life}\label{pandemic-and-just-the-general-nature-of-life}}

We are all going through difficult times during the pandemic. These include changes in our normal ways of doing things, living arrangements, family responsibilities, concerns about health and wellbeing, etc\ldots{} None of us is really OK all the time. If you tell me you're having trouble, I will not judge you or think less of you but will work with you to come up with a pathway forward. I hope you can extend me and the TA the same grace. You never owe me personal information about your health. You are always welcome to talk to me about things that you are going through. If I can't help you I might be able to direct you to someone who can. If you need extra help or need to miss a class or need more time please just ask. I will try to work with you.

  \bibliography{book.bib,packages.bib}

\end{document}
